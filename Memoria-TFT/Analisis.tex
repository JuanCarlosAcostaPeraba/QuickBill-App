%----------------------------------------------------------
% Capítulo 2 – Análisis
%----------------------------------------------------------

%----------------------------------------------------------
\section{Problema}
%----------------------------------------------------------
\begin{large}
El día a día de muchas pymes y profesionales autónomos, la facturación sigue un recorrido fragmentado que combina plantillas de hojas de cálculo, correos electrónicos y archivado físico de documentos. Cada paso requiere intervención manual para verificar importes, registrar pagos y mantener la trazabilidad, lo que incrementa la probabilidad de errores humanos y alarga el ciclo de cobro y pago. La ausencia de una plataforma única obliga a reintroducir datos en distintas aplicaciones de contabilidad, gestión de clientes y seguimiento de inventario, de modo que las inconsistencias se propagan y dificultan la elaboración de un estado financiero fiable.

La falta de conectividad estable agrava este panorama. Cuando el personal trabaja en movilidad o en ubicaciones con cobertura deficiente no puede emitir ni consultar facturas en tiempo real, por lo que la información se desactualiza y se generan cuellos de botella en la toma de decisiones. A ello se suman los riesgos inherentes al almacenamiento disperso, tanto en equipos locales como en servidores poco seguros, que exponen a las organizaciones a pérdidas de datos, accesos no autorizados y complicaciones durante las auditorías.

El marco regulador también introduce exigencias que resultan complejas de cumplir con herramientas tradicionales. La Ley 18 / 2022 de Creación y Crecimiento Empresarial obliga a emitir factura electrónica en todas las transacciones entre empresas, mantener un registro íntegro de los movimientos y garantizar la autenticidad de cada documento \cite{ley18_2022}. Para muchos negocios la adaptación implica invertir en nuevas soluciones, capacitar al personal y rediseñar los procesos internos, con el consiguiente impacto económico y operativo. La combinación de tareas manuales, carencia de integración y presión normativa describe un problema que no solo merma la productividad, sino que también compromete la competitividad y la capacidad de crecimiento de las empresas.
\end{large}

%----------------------------------------------------------
\section{Solución}
%----------------------------------------------------------
\begin{large}
Para resolver las ineficiencias detectadas se propone una plataforma de facturación concebida bajo el principio \textit{mobile first} y respaldada íntegramente por los servicios gestionados de Firebase, que actúan como fuente única de verdad para todos los datos. La solución incluye aplicaciones nativas para iOS y Android que comparten la misma lógica de negocio y un portal web administrativo dedicado al análisis financiero y a la configuración avanzada. Los tres clientes se conectan a Firestore mediante SDK nativos o, cuando procede, mediante peticiones REST, lo que garantiza la consistencia y simplifica el mantenimiento evolutivo.

Las aplicaciones móviles siguen un modelo \textit{offline-first}. Cuando el dispositivo carece de red, las facturas, los clientes o los productos se guardan en la caché local persistente cifrada; tan pronto como vuelve la conectividad, Firestore sincroniza los cambios con la nube y resuelve posibles conflictos mediante sellos temporales, de modo que la actividad puede continuar sin interrupciones.

El plano de servidor se apoya en Cloud Functions y Firestore, una base de datos de documentos con escalado automático. Esto ofrece alta disponibilidad sin administración de servidores y ajusta el coste al uso real, una ventaja decisiva para pymes y profesionales autónomos. Las funciones generan los PDF de las facturas, aplican la numeración secuencial exigida por la Ley 18/2022 y almacenan los archivos en Cloud Storage. Las reglas de seguridad de Firestore, combinadas con Firebase Authentication y el cifrado transparente de Google, protegen los datos en tránsito y en reposo, cumpliendo los requisitos del \gls{rgpd} y dejando un registro de auditoría para cada operación sensible.

El portal web, alojado en Firebase Hosting, muestra en tiempo real indicadores como ingresos cobrados, importes pendientes y ratio de morosidad. Estos paneles se actualizan al instante gracias a la suscripción en tiempo real a Firestore. La interfaz, construida con \gls{astro}, renderiza contenido estático en el servidor y aplica hidratación parcial solo donde es necesario, sigue las pautas WCAG 2.1 y se adapta a distintos tamaños de pantalla.

Este planteamiento elimina la fragmentación de herramientas, permite facturar desde cualquier lugar sin depender de la red, garantiza la trazabilidad legal y reduce drásticamente los costes operativos. El resultado es un flujo de trabajo unificado y escalable que minimiza errores y mejora la experiencia tanto de la empresa emisora como del cliente receptor.
\end{large}

%----------------------------------------------------------
\section{Metodología}
%----------------------------------------------------------
\begin{large}
El desarrollo de la plataforma se organiza mediante un enfoque ágil e iterativo incremental que combina un tablero Kanban con principios de Scrum. Las tareas se registran y priorizan en GitHub Projects; al limitar el trabajo en curso es posible concentrarse en entregas parciales funcionales y reajustar el alcance cuando resulta necesario. Tras una sesión inicial de planificación, en la que se definen los requisitos preliminares y el marco de trabajo, el proyecto se divide en cinco incrementos, y cada ciclo atraviesa cuatro fases bien diferenciadas.

Durante la fase de análisis se detallan los requisitos, las funcionalidades y los criterios de aceptación correspondientes a cada incremento. La fase de diseño establece la estructura de datos, las vistas y las interfaces mediante prototipos en Figma. Con el diseño definido, comienza la fase de implementación, donde se desarrolla la aplicación iOS en Xcode y la versión Android en Android Studio, se escriben Cloud Functions para Firebase en TypeScript y se construye el portal web en Astro, seleccionado por su capacidad para renderizar contenido estático y aplicar hidratación parcial únicamente cuando es necesario. Para garantizar la estabilidad, cada incremento concluye con una fase de pruebas manuales en dispositivos emulados.

Una vez finalizados los cinco ciclos se inicia la etapa dedicada a la elaboración de la documentación y a la preparación de la defensa del proyecto. En esta fase se actualizan los manuales de usuario, la guía de despliegue y la memoria técnica, y se generan capturas y métricas que respaldan los resultados obtenidos.

Este esquema incremental favorece una evolución progresiva del sistema, mantiene la calidad en cada entrega y proporciona flexibilidad para incorporar mejoras o responder a cambios surgidos a lo largo del desarrollo.
\end{large}

%----------------------------------------------------------
\section{Organización del proyecto}
%----------------------------------------------------------
\begin{large}
El desarrollo se ha estructurado en cinco fases sucesivas que abarcan desde la definición inicial de requisitos hasta el despliegue de la solución y las pruebas finales. Este planteamiento facilita la trazabilidad de las tareas, la detección temprana de riesgos y la medición objetiva del progreso.

\subsection*{Fase 1. Análisis de requisitos}
Durante esta etapa se recopilaron las necesidades de pymes y profesionales autónomos a fin de fijar los objetivos funcionales y priorizar las características de mayor impacto. El resultado fue un catálogo de historias de usuario con criterios de aceptación medibles y una hoja de ruta inicial.

\subsection*{Fase 2. Diseño}
Una vez definidos los requisitos, se trazó la arquitectura lógica del sistema y se elaboraron prototipos de interfaz en Figma. Los diagramas de entidad–relación y los flujos de navegación sirvieron para validar la coherencia de los datos y la usabilidad antes de escribir una sola línea de código.

\subsection*{Fase 3. Implementación móvil (iOS)}
El desarrollo comenzó con la aplicación para iOS, implementada en Swift mediante Xcode. En esta fase se incluyeron todas las funcionalidades clave: alta de empresas, generación de facturas, modo offline y sincronización con Firestore.

\subsection*{Fase 4. Implementación móvil (Android)}
A continuación se abordó la versión Android con Kotlin en Android Studio. Dado el calendario del proyecto, esta edición se centró en el núcleo funcional, es decir, la emisión y la consulta de facturas, reutilizando la lógica de negocio y los modelos de datos definidos para iOS.

\subsection*{Fase 5. Portal web y pruebas finales}
El ciclo concluyó con la construcción del portal administrativo en Astro, alojado en Firebase Hosting, y con la ejecución de pruebas de usabilidad, rendimiento y seguridad. Estas pruebas, realizadas en dispositivos emulados y en entornos reales, verificaron la integración entre las aplicaciones móviles, las Cloud Functions y la base de datos Firestore antes del lanzamiento definitivo.
\end{large}

%----------------------------------------------------------
\section{Herramientas}
%----------------------------------------------------------
\begin{large}
Esta sección describe las tecnologías empleadas en cada capa de la plataforma y las razones de su elección.

\subsection*{iOS}
Las aplicaciones para iPhone se implementan en \textbf{Swift}, un lenguaje moderno creado por Apple que prioriza la seguridad mediante tipado fuerte y gestión automática de memoria, y el rendimiento gracias a la optimización de sus compiladores \gls{llvm} \cite{swift_lang2025}.

El entorno de desarrollo es \textbf{Xcode} 16.3. Este IDE integra editor, depurador, simuladores de dispositivos y marcos de pruebas unitarias y de interfaz. Herramientas como Instruments permiten perfilar CPU, memoria y consumo energético, lo que facilita detectar cuellos de botella y mantener la fluidez de la experiencia. Con esta combinación el código se compila a binarios nativos y se alinea con las directrices de accesibilidad de iOS sin recurrir a SDK externos \cite{xcode16_3}.

\subsection*{Android}
La versión para Android se escribe en \textbf{Kotlin}, un lenguaje conciso y expresivo que incorpora nulabilidad segura y corrutinas para la programación asíncrona \cite{kotlin_lang2025}.

El desarrollo se realiza en \textbf{Android Studio} 2024.3.2 (Meerkat Feature Drop). Esta versión incorpora emuladores con perfiles personalizables, inspector de bases de datos y analizador de rendimiento en tiempo real, además de plantillas que agilizan la integración con bibliotecas Jetpack \cite{as2024_3_2}. La combinación de Kotlin con Android Studio reduce la verbosidad, mejora la legibilidad y disminuye la probabilidad de errores típicos de Java.

\subsection*{Web}
El portal administrativo se construye con \textbf{Astro}. Este framework genera HTML estático en la compilación y aplica hidratación parcial solo a los componentes que requieren interactividad, de modo que la carga inicial resulta ligera y accesible \cite{astro_docs2025}.

Para la gestión de dependencias se utiliza \textbf{pnpm}, un gestor que comparte una caché global y evita duplicados en \textit{node\_modules}, lo que acelera las instalaciones \cite{pnpm_docs2025}. El despliegue se realiza en Firebase Hosting y la lógica de servidor reside en Cloud Functions escritas en TypeScript, lo que proporciona escalado automático sin necesidad de administrar infraestructura.

\subsection*{Servicios en la nube}
La capa de backend y servicios gestionados se apoya en \textbf{Firebase}. Este conjunto de productos de Google aporta una base de datos NoSQL en tiempo real (Firestore), reglas de seguridad personalizables, autenticación multifactor y almacenamiento de archivos en Cloud Storage \cite{firebase_docs2025}. La lógica de servidor se desarrolla con Cloud Functions escritas en TypeScript, mientras que Firebase Hosting proporciona un despliegue estático con certificados TLS automáticos. Gracias a sus capacidades de escalado bajo demanda, la plataforma puede crecer sin intervención administrativa y sin inversiones iniciales en infraestructura.

\subsection*{Control de versiones}
El control de cambios se lleva a cabo con \textbf{Git}, sistema distribuido que mantiene un historial completo y permite revertir versiones con precisión \cite{git_book2023}. El repositorio se aloja en \textbf{GitHub}, plataforma que ofrece \textit{pull requests}, revisiones entre pares y flujos de integración y entrega continuas mediante GitHub Actions \cite{github_docs2025}. Cada confirmación ejecuta linters, pruebas y despliegues a canales de previsualización, lo que garantiza calidad continua y retroalimentación temprana.

\end{large}