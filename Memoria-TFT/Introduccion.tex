%----------------------------------------------------------
% Capítulo 1 – Introducción
%----------------------------------------------------------
\begin{large}
\epigraph{Si enciendes una lámpara para alguien, iluminas también tu propio camino.}{Buda}

La facturación constituye un pilar esencial de la actividad económica: certifica las transacciones comerciales, sustenta la trazabilidad fiscal y salvaguarda los derechos de las partes involucradas. A lo largo de la historia, desde las tablillas mesopotámicas hasta las facturas digitales que circulan hoy día, la tecnología ha perfeccionado este proceso para garantizar disponibilidad, autenticidad e integridad de la información \cite{origen_facturas}.

En España, la \textbf{Ley 18/2022 de Creación y Crecimiento Empresarial} ha acelerado esta transición al imponer la factura electrónica en todas las operaciones \gls{b2b}. Las grandes empresas ya la aplican de forma obligatoria y, para pymes y autónomos, la exigencia será plena a lo largo de 2025 \cite{ley18_2022}. Paralelamente, la expansión del trabajo remoto y la omnipresencia de los teléfonos móviles han evidenciado la brecha entre los programas de escritorio tradicionales y las necesidades de movilidad. Muchas soluciones \gls{saas} dependen de conectividad permanente o licencias costosas, generando fricciones en negocios que operan fuera de la oficina y en clientes que desean consultar sus facturas desde el móvil.

En este trabajo se aborda ese desafío mediante un sistema de facturación centrado en aplicaciones móviles nativas (iOS y Android) complementadas con un portal web administrativo.
\end{large}

%----------------------------------------------------------
\section{Motivación y antecedentes}
%----------------------------------------------------------

\begin{large}
La gestión de facturas en la pequeña y mediana empresa sigue apoyándose, en gran medida, en un conjunto de soluciones aisladas: sistemas \gls{erp} de sobremesa, hojas de cálculo improvisadas y repositorios documentales que raramente se comunican entre sí. Esta fragmentación provoca duplicidades de datos y dificulta la colaboración fluida entre empresario, personal interno y cliente final.

Al mismo tiempo, el auge de los servicios \gls{saas} ha desplazado el trabajo contable a la nube; sin embargo, muchos de estos portales exigen conectividad permanente para funcionar. En zonas con cobertura deficiente, la emisión o actualización de facturas se retrasa, lo que genera cuellos de botella y pone en riesgo la puntualidad de la información.

La presión regulatoria se intensifica. La \textbf{Ley 18/2022 de Creación y Crecimiento Empresarial} convierte la factura electrónica en un requisito obligatorio para todas las transacciones \gls{b2b}. Este cambio ya se aplica a las grandes corporaciones y entrará en vigor para pymes y autónomos a lo largo de 2025, lo que obliga a las organizaciones a modernizar sus flujos de trabajo y garantizar la trazabilidad completa del documento \cite{ley18_2022}.

Por último, la movilidad se ha consolidado como requisito operativo. Según el “2023 Mid-market Technology Trends Report” de Deloitte, la mayoría de las empresas de tamaño medio ofrece hoy acceso a sus plataformas de gestión desde dispositivos móviles, aunque con frecuencia esas aplicaciones replican solo las funciones básicas de escritorio \cite{estudio_movilidad}.
\end{large}

%----------------------------------------------------------
\section{Objetivos}
%----------------------------------------------------------
\begin{large}

\subsection{Objetivo general}
El objetivo principal de este trabajo consiste en desarrollar una plataforma de facturación integral que combine aplicaciones móviles nativas para iOS y Android con un portal web administrativo. Este sistema ofrecerá funcionamiento continuo en ausencia de conexión, cumplirá con la normativa vigente y facilitará una experiencia de usuario intuitiva y satisfactoria.

\subsection{Objetivos específicos}
El primer objetivo específico es diseñar una arquitectura de datos sólida que modele con claridad las entidades propias del dominio de facturación tales como empresas, usuarios y facturas y que permita ampliar y mantener el sistema con facilidad.

El segundo objetivo específico se centra en implementar un sistema de autenticación y autorización capaz de gestionar roles y permisos de manera rigurosa para garantizar que cada usuario acceda únicamente a las funcionalidades correspondientes a su perfil.

El tercer objetivo específico aborda el desarrollo de la lógica de generación automática de facturas incluyendo el cálculo adecuado de impuestos y la exportación de los documentos a formato PDF con calidad profesional y compatibilidad con estándares de facturación electrónica.

El cuarto objetivo específico persigue dotar al sistema de un modo de funcionamiento sin conexión que posibilite la creación y modificación de facturas en entornos offline así como la sincronización de los cambios con el servidor una vez restablecida la conectividad.

El quinto objetivo específico contempla la creación de un portal web administrativo que muestre cuadros de mando y métricas financieras detalladas con el fin de facilitar el seguimiento de ingresos, gastos y morosidad y apoyar la toma de decisiones.

El sexto objetivo específico consiste en asegurar la confidencialidad e integridad de la información mediante el uso de cifrado de datos en tránsito y en reposo para cumplir con los requisitos del \gls{rgpd}.
\end{large}

%----------------------------------------------------------
\section{Competencias específicas}
%----------------------------------------------------------
\begin{large}
El desarrollo de este proyecto ha exigido movilizar un conjunto de competencias técnicas directamente relacionadas con el diseño, implementación y evaluación de sistemas de información avanzados. A continuación se detallan las competencias específicas que se han puesto en práctica a lo largo de la realización de este trabajo:

\begin{itemize}
  \item \textbf{CI8.} Capacidad para analizar, diseñar, construir y mantener aplicaciones de forma robusta, segura y eficiente, seleccionando el paradigma y los lenguajes de programación más adecuados.  
  \item \textbf{CI12.} Conocimiento y aplicación de las características, funcionalidades y estructura de las bases de datos, que permitan su diseño, análisis e implementación de soluciones de almacenamiento de información.  
  \item \textbf{CI13.} Conocimiento y aplicación de las herramientas necesarias para el almacenamiento, procesamiento y acceso a los sistemas de información, incluidos los basados en web.  
  \item \textbf{CI16.} Conocimiento y aplicación de los principios, metodologías y ciclos de vida de la ingeniería del software, para asegurar procesos ordenados de desarrollo y mantenimiento.  
  \item \textbf{CI17.} Capacidad para diseñar y evaluar interfaces persona–computador que garanticen la accesibilidad y usabilidad de los sistemas, servicios y aplicaciones informáticas.  
  \item \textbf{TI3.} Capacidad para emplear metodologías centradas en el usuario y la organización en el desarrollo, evaluación y gestión de aplicaciones, asegurando la ergonomía y la adopción de las soluciones.  
  \item \textbf{TI6.} Capacidad de concebir sistemas, aplicaciones y servicios basados en tecnologías de red, incluyendo entornos web, comercio electrónico y computación móvil.
\end{itemize}

Estos conocimientos y habilidades, extraídos directamente del plan de estudios oficial, han sido fundamentales para llevar a cabo el análisis, diseño e implementación de la plataforma de facturación propuesta \cite{grado2024ulpgc}.
\end{large}

\pagebreak

%----------------------------------------------------------
\section{Alineamiento con los Objetivos de Desarrollo Sostenible}
%----------------------------------------------------------
\begin{table}[!htbp]
\caption{Grado de relación del TFT con los objetivos de desarrollo sostenible.}
\label{tab:ODS}
\centering
\begin{tabular}{|l|c|c|c|c|}
\cline{1-5}
 & \multicolumn{4}{c|}{Grado de relación} \\ \cline{2-5}
ODS & 0 No procede & 1 Bajo & 2 Medio & 3 Alto \\ \cline{1-5}
1 Fin de la Pobreza                & X &   &   &   \\ \cline{1-5}
2 Hambre cero                      & X &   &   &   \\ \cline{1-5}
3 Salud y Bienestar                & X &   &   &   \\ \cline{1-5}
4 Educación de calidad             & X &   &   &   \\ \cline{1-5}
5 Igualdad de género               & X &   &   &   \\ \cline{1-5}
6 Agua limpia y saneamiento        & X &   &   &   \\ \cline{1-5}
7 Energía asequible y no contaminante & X &   &   &   \\ \cline{1-5}
8 Trabajo decente y crecimiento económico & X &   &   &   \\ \cline{1-5}
9 Industria, innovación e infraestructuras &   &   &   & X \\ \cline{1-5}
10 Reducción de las desigualdades  &   & X &   &   \\ \cline{1-5}
11 Ciudades y comunidades sostenibles &   &   & X &   \\ \cline{1-5}
12 Producción y consumo responsables &   &   & X &   \\ \cline{1-5} 
13 Acción por el clima             &   & X &   &   \\ \cline{1-5}
14 Vida submarina                  & X &   &   &   \\ \cline{1-5}
15 Vida de ecosistemas terrestres  & X &   &   &   \\ \cline{1-5}
16 Paz, justicia e instituciones sólidas & X &   &   &   \\ \cline{1-5}
17 Alianzas para lograr objetivos  & X &   &   &   \\ \cline{1-5}
\end{tabular}
\end{table}

\begin{large}
La contribución de este proyecto a los Objetivos de Desarrollo Sostenible de la Agenda 2030 de Naciones Unidas se detalla a continuación \cite{un_agenda2030,sdg_report2023}.

En primer lugar, en relación con el \textbf{ODS 9 – Industria, innovación e infraestructuras}, la aplicación propone una infraestructura digital accesible y resiliente, capaz de operar sin conexión para garantizar la continuidad del negocio en entornos con conectividad limitada. Además, sus funciones de generación y exportación de facturas en PDF introducen innovaciones que modernizan los procesos administrativos. 

Respecto al \textbf{ODS 10 – Reducción de las desigualdades}, el diseño centrado en la usabilidad del teléfono móvil facilita que microempresas y autónomos, con recursos más reducidos, puedan adoptar tecnología avanzada de facturación, reduciendo la brecha digital frente a grandes organizaciones. 

En cuanto al \textbf{ODS 11 – Ciudades y comunidades sostenibles}, la eliminación del papel y la eliminación de desplazamientos para la entrega física de documentos contribuye a disminuir residuos urbanos y a apoyar la sostenibilidad económica y medioambiental de los comercios locales. 

El \textbf{ODS 12 – Producción y consumo responsables} se ve favorecido al sustituir procesos basados en facturas impresas por archivos cifrados en la nube, promoviendo prácticas \textit{paper-less} y un uso más eficiente de los recursos materiales y energéticos. 

Finalmente, el \textbf{ODS 13 – Acción por el clima} recibe un impulso gracias a la reducción del consumo de papel y de envíos postales, lo que disminuye la huella de carbono administrativa. Aunque el uso de dispositivos móviles y servicios en la nube conlleva consumo energético, el balance de emisiones es muy inferior al de los métodos tradicionales, siempre que se adopten buenas prácticas de eficiencia energética.
\end{large}