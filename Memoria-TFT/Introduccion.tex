%----------------------------------------------------------
% Capítulo 1 – Introducción
%----------------------------------------------------------
\begin{large}
\epigraph{Si enciendes una lámpara para alguien, iluminas también tu propio camino.}{Buda}

La facturación es uno de los pilares esenciales de la actividad económica: certifica las transacciones comerciales, sustenta la trazabilidad fiscal y salvaguarda los derechos de las partes involucradas. A lo largo de la historia, desde las tablillas mesopotámicas hasta las facturas digitales que circulan hoy día, la tecnología ha perfeccionado este proceso para garantizar disponibilidad, autenticidad e integridad de la información \cite{origen_facturas}.

En España, la \textbf{Ley 18/2022 de Creación y Crecimiento Empresarial} ha acelerado esta transición al imponer la factura electrónica en todas las operaciones \gls{b2b}. Las grandes empresas ya la aplican de forma obligatoria y, para pymes y autónomos, la exigencia será plena a lo largo de 2025 \cite{ley18_2022}. Paralelamente, la expansión del trabajo remoto y la omnipresencia de los teléfonos móviles han evidenciado la brecha entre los programas de escritorio tradicionales y las necesidades de movilidad. Muchas soluciones \gls{saas} dependen de conectividad permanente o licencias costosas, generando fricciones en negocios que operan fuera de la oficina y en clientes que desean consultar sus facturas desde el móvil.

En este trabajo se aborda ese desafío mediante un sistema de facturación centrado en aplicaciones móviles nativas (iOS y Android) complementadas con un portal web administrativo.
\end{large}

%----------------------------------------------------------
\section{Motivación y antecedentes}
%----------------------------------------------------------

\begin{large}
La gestión de facturas en la pequeña y mediana empresa sigue apoyándose, en gran medida, en un conjunto de soluciones aisladas: sistemas \gls{erp} de sobremesa, hojas de cálculo improvisadas y repositorios documentales que raramente se comunican entre sí. Esta fragmentación provoca duplicidades de datos y dificulta la colaboración fluida entre empresario, personal interno y cliente final.

Al mismo tiempo, el auge de los servicios \gls{saas} ha desplazado el trabajo contable a la nube; sin embargo, muchos de estos portales exigen conectividad permanente para funcionar. En zonas con cobertura deficiente, la emisión o actualización de facturas se retrasa, lo que genera cuellos de botella y pone en riesgo la puntualidad de la información.

La \textbf{Ley 18/2022 de Creación y Crecimiento Empresarial} convierte la factura electrónica en un requisito obligatorio para todas las transacciones \gls{b2b}. Este cambio ya se aplica a las grandes corporaciones y entrará en vigor para pymes y autónomos a lo largo de 2025, lo que obliga a las organizaciones a modernizar sus flujos de trabajo y garantizar la trazabilidad completa del documento \cite{ley18_2022}.

Por último, la movilidad se ha consolidado como requisito operativo. Según el “2023 Mid-market Technology Trends Report” de Deloitte, la mayoría de las empresas de tamaño medio ofrece hoy acceso a sus plataformas de gestión desde dispositivos móviles, aunque con frecuencia esas aplicaciones replican solo las funciones básicas de escritorio \cite{estudio_movilidad}.
\end{large}

%----------------------------------------------------------
\section{Objetivos}
%----------------------------------------------------------
\begin{large}

\subsection{Objetivo general}
El proyecto persigue la creación de una plataforma de facturación integral formada por aplicaciones móviles nativas para iOS y Android y por un portal web administrativo. La solución debe operar sin interrupciones, incluso en ausencia de conexión, y ofrecer una experiencia de uso clara, accesible y satisfactoria.

\subsection{Objetivos específicos}
En primer lugar se plantea el diseño de una arquitectura de datos sólida que represente con precisión las entidades del dominio, es decir, empresas, usuarios y facturas, y permita la evolución y el mantenimiento del sistema sin inconsistencias.

El segundo objetivo establece la implantación de un mecanismo de autenticación y autorización que gestione roles y permisos con rigor, de manera que cada perfil acceda únicamente a las funciones que le correspondan.

El tercer objetivo se centra en desarrollar la lógica de generación automática de facturas, incorporando el cálculo exacto de impuestos y la exportación de los documentos a PDF con calidad profesional y compatibilidad con los estándares de facturación electrónica.

El cuarto objetivo persigue proporcionar a la plataforma un modo de funcionamiento sin conexión que posibilite crear o modificar facturas en entornos offline y sincronizar los cambios en cuanto se restablezca la conectividad.

\pagebreak

El quinto objetivo aborda la construcción de un portal web administrativo con cuadros de mando y métricas financieras detalladas, cuyo propósito es facilitar el seguimiento de ingresos, gastos y morosidad y respaldar la toma de decisiones.

Por último, el sexto objetivo apunta a proteger la confidencialidad e integridad de la información mediante cifrado tanto en tránsito como en reposo, garantizando el cumplimiento de los requisitos establecidos por el \gls{rgpd}.
\end{large}

%----------------------------------------------------------
\section{Competencias específicas}
%----------------------------------------------------------
\begin{large}
El desarrollo de este proyecto ha exigido movilizar un conjunto de competencias técnicas directamente relacionadas con el diseño, implementación y evaluación de sistemas de información avanzados. A continuación se detallan las competencias específicas que se han puesto en práctica a lo largo de la realización de este trabajo:

\begin{itemize}[label=\raisebox{0.15ex}{\scriptsize$\blacksquare$}]
  \item \textbf{CI8.} Capacidad para analizar, diseñar, construir y mantener aplicaciones de forma robusta, segura y eficiente, seleccionando el paradigma y los lenguajes de programación más adecuados.  
  \item \textbf{CI12.} Conocimiento y aplicación de las características, funcionalidades y estructura de las bases de datos, que permitan su diseño, análisis e implementación de soluciones de almacenamiento de información.  
  \item \textbf{CI13.} Conocimiento y aplicación de las herramientas necesarias para el almacenamiento, procesamiento y acceso a los sistemas de información, incluidos los basados en web.  
  \item \textbf{CI16.} Conocimiento y aplicación de los principios, metodologías y ciclos de vida de la ingeniería del software, para asegurar procesos ordenados de desarrollo y mantenimiento.  
  \item \textbf{CI17.} Capacidad para diseñar y evaluar interfaces persona–computador que garanticen la accesibilidad y usabilidad de los sistemas, servicios y aplicaciones informáticas.  
  \item \textbf{TI3.} Capacidad para emplear metodologías centradas en el usuario y la organización en el desarrollo, evaluación y gestión de aplicaciones, asegurando la ergonomía y la adopción de las soluciones.  
  \item \textbf{TI6.} Capacidad de concebir sistemas, aplicaciones y servicios basados en tecnologías de red, incluyendo entornos web, comercio electrónico y computación móvil.
\end{itemize}

Estos conocimientos y habilidades, extraídos directamente del plan de estudios oficial, han sido fundamentales para llevar a cabo el análisis, diseño e implementación de la plataforma de facturación propuesta \cite{grado2024ulpgc}.
\end{large}

%----------------------------------------------------------
\section{Alineamiento con los Objetivos de Desarrollo Sostenible}
%----------------------------------------------------------
\begin{table}[H]
\caption{Grado de relación del TFT con los objetivos de desarrollo sostenible.}
\label{tab:ODS}
\centering
\begin{tabular}{|l|c|c|c|c|}
\cline{1-5}
 & \multicolumn{4}{c|}{Grado de relación} \\ \cline{2-5}
ODS & 0 No procede & 1 Bajo & 2 Medio & 3 Alto \\ \cline{1-5}
1 Fin de la Pobreza                & X &   &   &   \\ \cline{1-5}
2 Hambre cero                      & X &   &   &   \\ \cline{1-5}
3 Salud y Bienestar                & X &   &   &   \\ \cline{1-5}
4 Educación de calidad             & X &   &   &   \\ \cline{1-5}
5 Igualdad de género               & X &   &   &   \\ \cline{1-5}
6 Agua limpia y saneamiento        & X &   &   &   \\ \cline{1-5}
7 Energía asequible y no contaminante & X &   &   &   \\ \cline{1-5}
8 Trabajo decente y crecimiento económico & X &   &   &   \\ \cline{1-5}
9 Industria, innovación e infraestructuras &   &   &   & X \\ \cline{1-5}
10 Reducción de las desigualdades  &   & X &   &   \\ \cline{1-5}
11 Ciudades y comunidades sostenibles &   &   & X &   \\ \cline{1-5}
12 Producción y consumo responsables &   &   & X &   \\ \cline{1-5} 
13 Acción por el clima             &   & X &   &   \\ \cline{1-5}
14 Vida submarina                  & X &   &   &   \\ \cline{1-5}
15 Vida de ecosistemas terrestres  & X &   &   &   \\ \cline{1-5}
16 Paz, justicia e instituciones sólidas & X &   &   &   \\ \cline{1-5}
17 Alianzas para lograr objetivos  & X &   &   &   \\ \cline{1-5}
\end{tabular}
\end{table}

\begin{large}
En primer lugar, en relación con el \textbf{ODS 9 – Industria, innovación e infraestructuras}, la aplicación propone una infraestructura digital accesible, capaz de operar sin conexión para garantizar la continuidad del negocio en entornos con conectividad limitada. Además, sus funciones de generación y exportación de facturas en PDF modernizan los procesos administrativos. 

Respecto al \textbf{ODS 10 – Reducción de las desigualdades}, el diseño centrado en la usabilidad del teléfono móvil facilita que microempresas y autónomos, con recursos más reducidos, puedan adoptar tecnología avanzada de facturación, reduciendo la brecha digital frente a grandes organizaciones. 

En cuanto al \textbf{ODS 11 – Ciudades y comunidades sostenibles}, la eliminación del papel y la eliminación de desplazamientos para la entrega física de documentos contribuye a disminuir residuos urbanos y a apoyar la sostenibilidad económica y medioambiental de los comercios locales. 

El \textbf{ODS 12 – Producción y consumo responsables} se ve favorecido al sustituir procesos basados en facturas impresas por archivos cifrados en la nube, promoviendo prácticas \textit{paper-less} y un uso más eficiente de los recursos materiales y energéticos. 

Finalmente, el \textbf{ODS 13 – Acción por el clima} recibe un impulso gracias a la reducción del consumo de papel y de envíos postales, lo que disminuye la huella de carbono administrativa. Aunque el uso de dispositivos móviles y servicios en la nube conlleva consumo energético, el balance de emisiones es muy inferior al de los métodos tradicionales, siempre que se adopten buenas prácticas de eficiencia energética.
\end{large}